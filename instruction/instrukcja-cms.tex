\documentclass[12pt,a4paper]{article}
\usepackage[utf8]{inputenc}
\usepackage[T1]{fontenc}
\usepackage[polish]{babel}
\usepackage{graphicx}
\usepackage[hidelinks]{hyperref}
\usepackage{enumitem}
\usepackage{xcolor}
\usepackage{tcolorbox}
\usepackage{geometry}
\usepackage{titlesec}
\usepackage{fancyhdr}

\geometry{margin=2.5cm}

% Kolory
\definecolor{primary}{RGB}{59, 130, 246}
\definecolor{warning}{RGB}{245, 158, 11}
\definecolor{success}{RGB}{34, 197, 94}
\definecolor{info}{RGB}{99, 102, 241}

% Formatowanie nagłówków
\titleformat{\section}{\Large\bfseries\color{primary}}{\thesection.}{0.5em}{}
\titleformat{\subsection}{\large\bfseries}{\thesubsection.}{0.5em}{}
\titleformat{\subsubsection}{\normalsize\bfseries}{\thesubsubsection.}{0.5em}{}

% Stopka i nagłówek
\pagestyle{fancy}
\fancyhf{}
\fancyhead[L]{\small Instrukcja CMS -- Nie z tej bajki}
\fancyhead[R]{\small\thepage}
\fancyfoot[C]{\small \today}

% Ramki informacyjne
\newtcolorbox{uwaga}{
    colback=warning!10,
    colframe=warning,
    title=Uwaga,
    fonttitle=\bfseries
}

\newtcolorbox{wskazowka}{
    colback=success!10,
    colframe=success,
    title=Wskazówka,
    fonttitle=\bfseries
}

\newtcolorbox{informacja}{
    colback=info!10,
    colframe=info,
    title=Informacja,
    fonttitle=\bfseries
}

\begin{document}

% Strona tytułowa
\begin{titlepage}
    \centering
    \vspace*{3cm}
    {\Huge\bfseries Instrukcja obsługi\\[0.3cm] systemu CMS\par}
    \vspace{1cm}
    {\LARGE Strona internetowa\\[0.2cm] \textbf{Nie z tej bajki}\par}
    \vspace{2cm}
    {\large Przewodnik dla użytkownika\par}
    \vspace{3cm}
    {\large Wersja 1.0\par}
    \vfill
    {\large \today\par}
\end{titlepage}

\tableofcontents
\newpage

% ============================================================================
\section{Wprowadzenie}
% ============================================================================

\subsection{Czym jest CMS?}

CMS (ang. Content Management System) to system zarządzania treścią, który pozwala na łatwe edytowanie zawartości strony internetowej bez znajomości programowania. Dzięki niemu możesz:

\begin{itemize}
    \item Dodawać i edytować teksty na stronie
    \item Wgrywać zdjęcia i dokumenty
    \item Tworzyć nowe aktualności i wpisy
    \item Zarządzać galerią zdjęć
    \item Aktualizować informacje o kadrze
\end{itemize}

\subsection{Jak zmiany w CMS wpływają na stronę?}

Kiedy wprowadzasz zmiany w panelu CMS, nie są one od razu widoczne na stronie internetowej. Aby zmiany zostały opublikowane, musisz \textbf{wdrożyć stronę} (przebudować ją). Proces ten zajmuje kilka minut.

\begin{informacja}
Możesz spokojnie pracować nad treściami w CMS -- zmiany nie pojawią się na stronie, dopóki nie klikniesz przycisku ``Przebuduj stronę'' w sekcji \textbf{Wdrażanie}.
\end{informacja}

% ============================================================================
\section{Logowanie i nawigacja}
% ============================================================================

\subsection{Logowanie do panelu CMS}

\begin{enumerate}
    \item Otwórz przeglądarkę internetową (np. Chrome, Firefox)
    \item Wejdź na adres panelu CMS (otrzymasz go od administratora)
    \item Zaloguj się używając swojego konta
\end{enumerate}

\subsection{Struktura panelu}

Po zalogowaniu zobaczysz panel z menu bocznym po lewej stronie. Menu zawiera następujące sekcje:

\begin{description}[style=nextline]
    \item[Strony] Edycja stałych podstron (Szkoła, Placówka)
    \item[Globalne] Ustawienia całej strony (dane kontaktowe, logo)
    \item[Dokumenty] Treści dynamiczne (aktualności, kluby, galerie, cytaty, partnerzy)
    \item[Wdrażanie] Publikowanie zmian na stronę
\end{description}

% ============================================================================
\section{Zarządzanie stronami}
% ============================================================================

W sekcji \textbf{Strony} znajdziesz dwie podstrony do edycji:

\subsection{Strona Szkoły}

Na tej stronie możesz zarządzać:

\subsubsection{Kadrą szkoły}

Sekcja ``Zdjęcia kadry'' pozwala dodawać członków kadry pedagogicznej. Dla każdej osoby lub grupy osób możesz dodać:

\begin{itemize}
    \item \textbf{Pierwsze zdjęcie} -- główne zdjęcie (wymagane)
    \item \textbf{Drugie zdjęcie} -- zdjęcie alternatywne, np. w innej pozie (wymagane)
    \item \textbf{Imiona i nazwiska} -- lista osób na zdjęciu z ich stanowiskami
\end{itemize}

\begin{wskazowka}
Dwa zdjęcia są potrzebne, ponieważ na stronie wyświetlany jest efekt zmiany zdjęcia po najechaniu myszką.
\end{wskazowka}

\subsubsection{Dokumentami do pobrania}

W tej sekcji możesz dodawać pliki PDF (regulaminy, statuty, dokumenty), które odwiedzający będą mogli pobrać ze strony.

Dla każdego pliku podaj:
\begin{itemize}
    \item \textbf{Nazwa pliku} -- czytelna nazwa, np. ``Statut szkoły 2024''
    \item \textbf{Plik} -- wgraj dokument PDF
\end{itemize}

\subsection{Strona Placówki}

Strona placówki zawiera tylko sekcję \textbf{Dokumenty do pobrania} -- działa identycznie jak w przypadku szkoły.

% ============================================================================
\section{Zarządzanie dokumentami}
% ============================================================================

Sekcja \textbf{Dokumenty} zawiera treści, które można wielokrotnie dodawać (w przeciwieństwie do stron, które są pojedyncze).

\subsection{Aktualności}

Aktualności to wpisy blogowe/newsowe wyświetlane na stronie w sekcji ``Projekty/Aktualności''.

\subsubsection{Tworzenie nowej aktualności}

\begin{enumerate}
    \item Kliknij \textbf{Dokumenty} $\rightarrow$ \textbf{Aktualności}
    \item Kliknij przycisk \textbf{+ Utwórz} (w prawym górnym rogu)
    \item Wypełnij formularz:
\end{enumerate}

\begin{description}[style=nextline]
    \item[Tytuł] Nagłówek aktualności (wymagany)
    \item[Data publikacji] Data wyświetlana przy wpisie (wymagana)
    \item[Krótki opis] Tekst widoczny na liście aktualności (wymagany)
    \item[Partner] Opcjonalnie -- jeśli aktualność dotyczy partnera, wybierz go z listy. Jego logo zostanie wyświetlone jako zdjęcie aktualności.
    \item[Zdjęcie] Jeśli nie wybrano partnera -- wgraj zdjęcie aktualności
    \item[Opis] Pełna treść aktualności (edytor tekstu)
    \item[Slug] Adres URL aktualności (generowany automatycznie z tytułu)
\end{description}

\subsubsection{Edytor tekstu (Opis)}

Pole ``Opis'' to zaawansowany edytor tekstu, który pozwala na:

\begin{itemize}
    \item Formatowanie tekstu (pogrubienie, kursywa, podkreślenie)
    \item Tworzenie nagłówków (różne poziomy)
    \item Tworzenie list (punktowanych i numerowanych)
    \item Dodawanie cytatów
    \item Wstawianie linków do innych stron
    \item \textbf{Wstawianie zdjęć} -- pojedynczych lub w formie siatki
    \item \textbf{Wstawianie filmów} -- z YouTube lub Vimeo
\end{itemize}

\begin{uwaga}
Aby wstawić film, skopiuj link z YouTube lub Vimeo i wklej go w polu ``Link do filmu''. Link musi być pełnym adresem strony z filmem.
\end{uwaga}

\subsection{Kluby}

Kluby działają bardzo podobnie do aktualności. Służą do prezentacji kółek zainteresowań i klubów działających przy placówce.

Pola są identyczne jak w aktualnościach, z tą różnicą, że:
\begin{itemize}
    \item Wpisy wyświetlane są w osobnej sekcji ``Kluby'' na stronie
    \item Możesz powiązać klub z partnerem (np. organizacją prowadzącą)
\end{itemize}

\subsection{Galeria -- Foldery}

Galeria pozwala tworzyć albumy zdjęć pogrupowane w foldery.

\subsubsection{Tworzenie nowego folderu galerii}

\begin{enumerate}
    \item Kliknij \textbf{Dokumenty} $\rightarrow$ \textbf{Galeria - Foldery}
    \item Kliknij \textbf{+ Utwórz}
    \item Wypełnij:
\end{enumerate}

\begin{description}[style=nextline]
    \item[Nazwa folderu] Tytuł albumu, np. ``Wycieczka do zoo 2024'' (wymagana)
    \item[Okładka] Zdjęcie wyświetlane jako miniatura folderu (wymagane)
    \item[Slug] Adres URL folderu (generowany automatycznie)
    \item[Zdjęcia] Lista wszystkich zdjęć w folderze
\end{description}

\begin{wskazowka}
Możesz dodać wiele zdjęć na raz, przeciągając je z komputera do pola ``Zdjęcia''.
\end{wskazowka}

\subsubsection{Tekst alternatywny zdjęć}

Dla każdego zdjęcia należy podać \textbf{tekst alternatywny} (pole ``Alt''). Jest to krótki opis tego, co przedstawia zdjęcie. Tekst ten:

\begin{itemize}
    \item Jest czytany przez programy dla osób niewidomych
    \item Wyświetla się, gdy zdjęcie nie może się załadować
    \item Pomaga w pozycjonowaniu strony w Google
\end{itemize}

Przykłady dobrych tekstów alternatywnych:
\begin{itemize}
    \item ``Dzieci podczas zabawy na placu zabaw''
    \item ``Występ teatralny uczniów klasy 3a''
    \item ``Widok na budynek szkoły od frontu''
\end{itemize}

\subsection{Cytaty}

Cytaty wyświetlane są na stronie głównej w formie karuzeli.

\begin{description}[style=nextline]
    \item[Treść cytatu] Tekst cytatu (max. 280 znaków)
    \item[Autor] Osoba, której przypisany jest cytat
\end{description}

\begin{informacja}
Cytaty są krótkie -- jeśli przekroczysz 280 znaków, system wyświetli ostrzeżenie.
\end{informacja}

\subsection{Partnerzy}

Partnerzy to organizacje współpracujące z placówką. Ich loga wyświetlane są na stronie głównej.

\begin{description}[style=nextline]
    \item[Nazwa partnera] Pełna nazwa organizacji (wymagana)
    \item[Logo/Zdjęcie] Grafika reprezentująca partnera (wymagane)
    \item[Slug] Adres URL (generowany automatycznie)
\end{description}

Po dodaniu partnera możesz go wybierać przy tworzeniu aktualności i klubów -- wtedy logo partnera zostanie automatycznie użyte jako zdjęcie wpisu.

% ============================================================================
\section{Ustawienia globalne}
% ============================================================================

W sekcji \textbf{Globalne} $\rightarrow$ \textbf{Ustawienia Strony} znajdziesz opcje dotyczące całej witryny.

\subsection{Zakładka ``Podstawowe''}

\begin{description}[style=nextline]
    \item[Tytuł strony] Główny tytuł wyświetlany w przeglądarce
    \item[Opis strony] Domyślny opis dla wyszukiwarek
    \item[Adres URL strony] Główny adres witryny
    \item[Favicon] Mała ikonka wyświetlana w karcie przeglądarki
\end{description}

\subsection{Zakładka ``Organizacja''}

Dane kontaktowe organizacji wyświetlane na stronie:

\begin{description}[style=nextline]
    \item[Nazwa organizacji] Pełna nazwa stowarzyszenia/placówki
    \item[Logo] Logo wyświetlane w stopce
    \item[Adres] Pełny adres siedziby
    \item[Telefon] Numer kontaktowy
    \item[Email] Adres e-mail kontaktowy
\end{description}

\subsection{Zakładka ``Media społecznościowe''}

Linki do profili w mediach społecznościowych:
\begin{itemize}
    \item Facebook
    \item Instagram
    \item LinkedIn
    \item Twitter/X
\end{itemize}

\begin{uwaga}
Wpisuj pełne adresy URL, np. \texttt{https://facebook.com/nazwastrony}
\end{uwaga}

% ============================================================================
\section{Optymalizacja dla wyszukiwarek (SEO)}
% ============================================================================

Większość treści w CMS posiada zakładkę \textbf{SEO}. Wypełnienie tych pól pomoże Twojej stronie lepiej wyświetlać się w wynikach wyszukiwania Google.

\subsection{Podstawowe pola SEO}

\begin{description}[style=nextline]
    \item[Tytuł meta] Tytuł wyświetlany w wynikach Google (max. 60 znaków)
    \item[Opis meta] Opis pod tytułem w Google (max. 160 znaków)
    \item[Słowa kluczowe] Frazy, po których strona ma być znajdowana
\end{description}

\subsection{Pola Open Graph}

Pola Open Graph kontrolują, jak strona wygląda po udostępnieniu na Facebooku lub w innych mediach społecznościowych:

\begin{description}[style=nextline]
    \item[Obraz Open Graph] Zdjęcie wyświetlane przy udostępnieniu (zalecane 1200x630 px)
    \item[Tytuł Open Graph] Tytuł przy udostępnieniu (jeśli inny niż meta)
    \item[Opis Open Graph] Opis przy udostępnieniu
\end{description}

\subsection{Pozostałe opcje}

\begin{description}[style=nextline]
    \item[Nie indeksuj] Zaznacz, jeśli strona nie ma się pojawiać w Google
    \item[Kanoniczny URL] Używane przy duplikatach treści (zazwyczaj zostaw puste)
\end{description}

\begin{wskazowka}
Jeśli nie wiesz, co wpisać w polach SEO -- możesz je zostawić puste. System użyje wartości domyślnych z ustawień globalnych.
\end{wskazowka}

% ============================================================================
\section{Wdrażanie zmian}
% ============================================================================

Po wprowadzeniu wszystkich zmian w CMS, musisz opublikować je na stronie.

\subsection{Jak wdrożyć zmiany?}

\begin{enumerate}
    \item Upewnij się, że wszystkie zmiany są zapisane
    \item Kliknij \textbf{Wdrażanie} w menu bocznym
    \item Kliknij przycisk \textbf{Przebuduj stronę}
    \item Potwierdź operację w oknie dialogowym
    \item Poczekaj na komunikat o powodzeniu
\end{enumerate}

\begin{informacja}
Przebudowa strony trwa zazwyczaj 2-5 minut. Po tym czasie zmiany będą widoczne na stronie internetowej.
\end{informacja}

\begin{uwaga}
Nie musisz wdrażać strony po każdej małej zmianie. Możesz wprowadzić wiele zmian, a następnie wdrożyć je wszystkie naraz.
\end{uwaga}

\end{document}
